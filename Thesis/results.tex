\chapter{Results}
\section{Analysis}

The experiments clearly showed first a relationship between cell state of charge the acoustic TOF through the cell, between induced lithium plating and changes in acoustic time of flight data, and finally between changes in ambient temperature and change in acoustic TOF through the cell. 

Recall that the three sources of change in acoustic time of flight through a cell are change in state of charge, change in state of health, and change in ambient conditions. 
By systematically adding and understanding changes in each category to the experimental conditions, shifts in acoustic TOF through the cell can be observed and even attributed to each source of change.

When the ambient conditions were held steady, the relationship was very clear (insert data). Ambient temperature variation clearly has a significant effect on the acoustic TOF shifts. \todo{fig} shows a comparison between the measured TOF data and "idealized" TOF from a gentle, nonplating routine, constructed by assuming the data ought not change between cycles and copying the first waveform repeatedly as a result. In the absence of plating and at the same state of charge, the relationship between change in TOF shift and change in ambient temperature is consistent enough that much of the influence of temperature can be scrubbed out using simple regressions \todo{fig}.

\subsection{Time of Flight Shift Due to Change in State of Charge}

\subsection{Time of Flight Shift Due to Change in State of Health}

\subsection{Time of Flight Shift Due to Change in Ambient Conditions}