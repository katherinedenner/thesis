High performance batteries such as those critical to electrical vehicle adoption require constant, accurate estimation of many cells' state of charge and state of health, and deft management of the battery based on that information is critical to both vehicle performance and safety.
Inadequacies in the state of the art are holding back more widespread electric vehicle adoption.

This thesis builds on prior work in developing new, ultrasonic methods for measuring battery state of charge and state of health in operando, and for the first time demonstrates the technology's ability to measure ambient temperature, a key metric for achieving safe, high performance electric vehicles. Prior work targeted battery formats such as AA. The fast-charge batteries favored by high-performance applications present additional challenges due to their rapidly-shifting properties and compact form factors.

Described within is the design of an apparatus to electrically manipulate the cell's SoC and SoH while regulating its ambient temperature and pressure, the procedures to explore the validity of acoustic ToF to fast-charge cells, and the analysis to synthesize and interpret acoustical, electrochemical, and thermal data.

This thesis demonstrates the capability of acoustic time of flight analysis to observe changes in the state of charge and state of health in a fast-charge lithium ion cell, and for the first time, to observe changes in the cell's ambient temperature.
