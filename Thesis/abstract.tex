%%
%% MOTIVATION
%%
High performance batteries such as those critical to electrical vehicle adoption require constant, accurate estimation of many cells' state of charge and state of health, and deft management of the battery based on that information is critical both to vehicle performance and safety. 
Inadequacies in the state of the art are holding back more widespread electric vehicle adoption. 

This thesis builds on prior work in developing new, ultrasonic methods for measuring battery state of charge and state of health in operando, and for the first time demonstrates the technology's ability to measure ambient temperature, a key metric for both performance and safety in electric vehicles.
Prior work in this area targeted standard consumer battery formats such as AA. 
The fast-charge batteries favored by high-performance applications such as electric vehicles present additional challenges due to their more rapidly-shifting properties and more compact form factors, yet also present greater opportunity because so much more can be gained by improving their performance. 
The work herein focuses on extreme fast-charge cells, but its results can be generalized to other cells which are easier to work with.

Described within is the design of an experimental apparatus was designed tp allow for electrically manipulating the lithium-ion cell's state of charge and state of health while regulating its ambient temperature and pressure; the experimental procedures used to explore the validity of these acoustic time of flight techniques to lithium-ion fast-charge cells; and the data analysis used to synthesize and interpret acoustical, electrochemical, and thermal data.

Ultimately, this thesis demonstrates the capability of acoustic time of flight analysis to observe changes in the state of charge and state of health in a fast-charge lithium-ion cell, and for the first time, to observe changes in the cell's ambient temperature.