\chapter{Conclusion}

This thesis demonstrated the capability of acoustic time of flight analysis to observe changes in the state of charge and state of health in a fast-charge lithium-ion cell, and even to observe changes in the cell's ambient temperature. 

First, appropriate background was provided. Then, the requirements for and design of an experimental apparatus were provided. 
In short, this apparatus allows for the control of the three factors in shifts in time of flight of an acoustic wave through the lithium-ion cell: ambient conditions (in this case, temperature and pressure are of interest), cell state of charge, and cell state of health. 
The apparatus allows for the measurement and control of ambient temperature and pressure around the cell, and electrical load on the cell. 
Experimental procedures were established for observing and at times manipulating the electrochemical, acoustical, and thermal state of the cell and its environment. Finally, the data gathered while following those procedures was analyzed.

Ultimately, the analysis demonstrated the validity of acoustic time-of-flight analysis for observing changes in state of charge, state of health, and even ambient conditions in lithium-ion fast-charge cells. \todo{accomplished?} Further work could include going deeper into this line of research by building predictive models to quantify observed plating, developing an experimental apparatus and protocol that would allow for probing cells stacked in a more commercially-typical manner, or adapting the apparatus and analysis to work with cheaper, lower frequency sensors.