\chapter{Conclusion}
This thesis builds on prior work in developing new, ultrasonic methods for measuring battery state of charge and state of health \emph{in operando}, and for the first time demonstrates the technology's ability to measure ambient temperature, a key metric for achieving safe, high performance electric vehicles. This work was able to demonstrate for the first time that ultrasonic state of charge and state of health monitoring methods previously developed for less finicky battery chemistries and form factors is applicable even to extreme fast-charge lithium ion cells, despite their size being on the order of microns, which greatly increased the influence of ambient effects. The work demonstrated that this could actually be viewed as a feature, as a clear relationship was demonstrated between temperature shift and time of flight shift. It was shown that this relationship is numerically consistent for a particular cell, and can even be used to reasonably predict what the readings would have been had temperature not shifted. 
Once a cell was known, this relationship could be reversed, allowing the same sensor to be used for measuring state of charge, state of health, and temperature. The added ability for a sensor to also measure temperature could be very useful in high-performance applications such as electric vehicles, which rely on careful temperature monitoring to ensure safe battery operation and long battery lifetime.
Further work could include going deeper into this line of research by building predictive models to quantify observed plating, developing an experimental apparatus and protocol that would allow for probing cells stacked in a more commercially-typical manner, or adapting the apparatus and analysis to work with cheaper, lower frequency sensors.