\section{Motivation}
Increasing demand for electric vehicles is pushing the need for extreme fast charge capabilities in lithium battery chemistries, and a critical limiting factor in charging rates is lithium deposition \cite{XFC}. Charging a lithium battery too quickly causes lithium deposition, which reduces cell capacity and in severe cases, creates safety hazards \cite{XFC}. Currently there is no practical, low-cost in operando method of detecting lithium deposition, or even changes in SoH and SoC \cite{SOC-SOH-EST}. Without real-time lithium deposition monitoring, safety and financial considerations require batteries with lithium-based cells to use controllers programmed to charge and discharge conservatively. Being able to monitor lithium plating would allow battery management systems for lithium cells to more accurately control charging rates, and potentially even employ charge and discharge protocols that unplate lithium. Such functionality would enable electric vehicles  to have both better performance and better safety.

The objective of this thesis is to use ultrasonic methods to demonstrate the feasibility of detecting the deposition of lithium and determining changes in the state of charge and state of health of single layer, high-rate lithium ion cells. Such techniques have been applied before to cells of other chemistries and form factors \cite{SOC-SOH-EST}, \cite{TOF-STATE}, \cite{ANODE-CHAR}, but not yet the sort of tricky fast-charge cells required by demanding applications such as electric vehicles.