% lithium ion: lithium cobalt oxide / graphite

\chapter{Experimental procedure}

\section{Observing ToF Shifts Due to Change in Charge State}

\subsection{Background}
\todo{need?}

\subsection{ToF Analysis of Cell at Rest in Uncontrolled Environment}

The cell was cycled (charged and discharged) five times at a 1/2C rate. First, the cell would rest, so it and its transducers could acclimate to its environment (the transducers are also affected by shifts in temperature). Then, the cell charges at a constant-current rate of 0.0120 Amps. The charging cycle was constant-current rather than constant-voltage \todo{because}. The 1/2C rate was chosen \todo{because}. The cycle then rested ten minutes, was discharged using the same constant-current, 1/2C protocol, and then rested for ten minutes again. The rests were \todo{because}. Then, the rest-charge-rest-discharge sequence was repeated for a total of five cycles.

\todo{insert flowchart}

Unfortunately, there was too much noise in the readings due to environmental interference, ultimately drowning the acoustic ToF signal. At that point, the source of the noise was unknown.

\todo{section title}
\todo{insert plots?}
    
\subsection{Acoustic observation of Cell at Rest in Thermal Chamber}

Since it was impossible to determine the source of changes in ToF data for the prior test, or to determine what was signal and what was noise, a more peicewise approach to building up the experiment was needed. Due to the very small size of the cell, small variations in temperature and pressure could potentially have relatively large impacts on the speed with which high-frequency acoustic waves travel through the cell. Some possible solutions were identified as regulating the temperature to be relatively constant, better regulating the pressure applied to the cell, and using higher frequency transducers to probe the cell. Switching the 2.5MHz transducers for 15MHz transducers was quick and easy since the second set of transducers was already on hand, so this was done for the next trial. Additionally, the relatively loosely fluctuating ambient temperature was expected to have a relatively large effect on time of flight, so the pressurizing apparatus was inserted into a thermal chamber, as described in \todo{rel sec}. The stack pressure is already expected to be constant to within \todo{insert estimate}, so this was left unchanged, to be revisited only if necessary.

Additionally, the cycling was postponed to a later trial. For now, the cell would be left at rest. 
If there was no significant shift in acoustic ToF data while the cell waited at rest for a long period, then the noise in the environment is adequately controlled. 
If there were still shifts, the environment would need further control, probably through active temperature control and higher-precision, active pressure control. 
For 24 hours, the cell was probed ultrasonically while pressured in the thermal chamber. \todo{thermal, pressure data}.

\todo{plots}
The above data show that the ToF did not shift while the temperature was held constant, and then shifted in step with a shift in temperature at the end, verifying that this setup effectively reduced environmental noise to acceptable levels to begin attempting to observe ToF shifts due to cell state.
\todo{describe plots a bit}

    
\subsection{Acoustic observation of cycling cell in thermal chamber}
\todo{insert plots?}
Once the noise levels were brought under control, attempting to observe change of state via acoustic ToF shift could begin. The cycle was set up as in \todo{rel sec}, with a long initial rest period to ensure starting conditions were steady state, then five cycles of a 10 minute rest -> 1/2C current-controlled charge -> 10 minute rest -> 1/2C current-controlled discharge; see \todo{fig}. In total this took over 26 hours. The temperature was kept constant at about \todo{number}, and the system was kept pressurized to \todo{number}, which was a stack pressure of about \todo{number}, falling into Canarella's \todo{which} regime.

\section{Observing ToF Shifts Due to Lithium Plating} 
Once confirmed that this set-up was capable of observing change of state of charge of the cell, it could be investigated whether it could observe changes in state of health.

\subsection{Background}
\todo{need?}

\subsection{Attempting to Acoustically Observe Lithium Plating of Cycling Cell in Thermal Chamber}
\todo{flowchart}
The battery was set-up in the pressurized, thermally-regulated apparatus as before; this time at a temperature of \todo{temp} C and a system pressurization of \todo{pres} psi, which applied a stack pressure of \todo{stack}. 
The cell was again cycled but this time with aggressive ramps in charging rate. 
This intentionally reckless operation would force significant lithium plating to develop on the anode \todo{yea?}. 
After a rest to ensure steady-state conditions, the cell was given a ten minutes rest, charged at a 1C rate for 60 minutes, given 10 minutes rest, and discharged at a C/2 rate. 
This sequence was repeated once, for two total cycles. 
It was then repeated with a charging protocol of 2C for 30 minutes, then 5C for 12 minutes, then 10C for 6 minutes. 
Lithium plating was expected to be observed in the latter two cycles \todo{why}.
\todo{insert plots?}
    
\subsubsection{X-Ray Photoelectron Spectroscopy Analysis of Cell}
    To independently determine the extent of lithium plating in the cell, it was analyzed via X-Ray Photoelectron Spectroscopy.

\subsection{Attempt to Force and Observe Lithium Unplating}

\subsubsection{XPS Analysis of Cell}

\subsubsection{SEM Analysis of Cell}