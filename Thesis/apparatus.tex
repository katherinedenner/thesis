\chapter{Apparatus design}

\section{Considerations}
\begin{enumerate}
    \item Frequency: resolution vs attenuation
    \begin{enumerate}
        \item Thinner cells require shorter wave lengths for useful resolution
	    \item Shorter wavelengths require higher frequencies
		\item Higher frequencies introduce greater attenuation
	\end{enumerate}
	\item Set-up: avoid near-field effects without introducing significant impedance
		\begin{enumerate} 
		    \item impedance is the result of medium transitions (e.g. the light traveling through air into a pool of water problem from physics)'
		    \item The near-field effects make the earliest readings highly unwieldy
	        \item Need to develop an apparatus that places enough distance between the transducer and the cell that the near-field effects have fallen away, but does so without introducing significant new impedances - be sure to impedance match materials and minimize number of materials
	    \end{enumerate}
	\item Pressure vs performance
	    \begin{enumerate}
		\item Need to exert enough pressure on the cell to ensure quality, consistent readings, but not so much that it materially effects battery behavior
		\item Should be adjustable
		\item Should apply a consistent pressure (<=0.5MPa) for SOC/SOH reasons, preferably light (to reduce experiment rest times) but potentially heavy if additional lithium deposition needs to be forced (see Cannarella)
		\end{enumerate}
    \item Securely and planarly mount as many as three transducers (to compare readings from different regions of cell - deposition should be asymmetric since leads are on one side)
    \item Avoid clamping to the edges: the lamination sometimes causes lil bulges
\end{enumerate}

Questions:
\begin{enumerate}
	\item How to couple transducer to cell?
	\item How to clamp cell?
	\item How to secure transducers?
    \item How to actuate set-up?
\end{enumerate}

\section{Sensors}

\section{Structure}

\section{Actuator}