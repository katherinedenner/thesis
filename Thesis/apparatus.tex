\chapter{Apparatus design}

\section{Considerations}
\begin{enumerate}
    \item Frequency: resolution vs attenuation
    \begin{enumerate}
        \item Thinner cells require shorter wave lengths for useful resolution
	    \item Shorter wavelengths require higher frequencies
		\item Higher frequencies introduce greater attenuation
	\end{enumerate}
	\item Set-up: avoid near-field effects without introducing significant impedance
		\begin{enumerate} 
		    \item impedance is the result of medium transitions (e.g. the light traveling through air into a pool of water problem from physics)'
		    \item The near-field effects make the earliest readings highly unwieldy
	        \item Need to develop an apparatus that places enough distance between the transducer and the cell that the near-field effects have fallen away, but does so without introducing significant new impedances - be sure to impedance match materials and minimize number of materials
	    \end{enumerate}
	\item Pressure vs performance
	    \begin{enumerate}
		\item Need to exert enough pressure on the cell to ensure quality, consistent readings, but not so much that it materially effects battery behavior
		\item Should be adjustable
		\item Should apply a consistent pressure (<=0.5MPa) for SOC/SOH reasons, preferably light (to reduce experiment rest times) but potentially heavy if additional lithium deposition needs to be forced (see Cannarella)
		\end{enumerate}
    \item Securely and planarly mount as many as three transducers (to compare readings from different regions of cell - deposition should be asymmetric since leads are on one side)
    \item Avoid clamping to the edges: the lamination sometimes causes lil bulges
\end{enumerate}

\section{Sensors}

\section{Structure}
To carry out this experiment, the apparatus must secure the cell for the duration of the testing while reliably applying the desired stack pressure, must keep the ultasonic transducers on each side of the cell aligned (coaxial) with each other, must align the transducers perpendicular to the cell, and must seperate the cell and the transducers sufficiently to reduce noise; each of these requirements will be discussed in detail soon.

The basic apparatus is a pair of blocks mounted to a rail. A row of holes is drilled into the blocks to hold an array of ultrasonic transducers, with channels milled out for routing their cables. A small plate with its own set of holes and channels bolts to this block to hold the transducers in place. Bolting to this plate is a small cylinder of Rexolite (a nylon variant), used as a low-interference medium for manipulating the travel distance of the ultrasonic waves to eliminate near-field effects, as will be discussed in greater detail later.

Behind the first set of holes in the block is a second set of narrower holes, for spring-loading the transducers. The springs ensure the transducers are pressed against the face of the transfer medium. The faces of the rexolite cylinders are turned flat so that the transducer emits and receives waves at an angle perpendicular to the test article. The plate further facilitates keeping everything coaxial. The three parts bolts together using one set of bolts which pass through all three, ensuring a tight, flat fit. This both keeps the transducers perpendicular to the test article and coaxial with each each other, as well as preventing any disruptive air from getting in the path of the ultrasonic waves. This would be bad because \TODO{explain}. Aluminum was chosen as the material for the blocks for its durability and machinability. Machinability was important because of the relatively tight tolerances required of these parts. Since the test articles are so thin and the ultrasonic transducers to high-frequency, small differences in actual versus expected transducer position could create insurmountable noise in the readings. The total apparatus has two of these block assemblies. One block is bolted directly to the rail and held stationary during the tests. The other is connected to via threaded rod to a double-acting piston and fixed to a carriage which rides along the rail. The piston is fixed to the rail by a bracket, and extends to push the block along the rail, securing the cell. By varying the pressure supplied to the piston, it can be used to apply a controlled stack pressure to the cell during testing.

The rexolite cylinder is included to ensure that the ultrasonic waves experience Fraunhofer (far field) diffraction, rather than noisy Fresnel (near field) diffraction. \todo{do I need to cite this section?}
Fraunhofer diffraction occurs when
$$ \frac{W^2}{L\lambda} << 1$$
where $W$ is the aperture or slit size, $L$ is the distance from the aperature, and $\lambda$ is the wavelength.
$\lambda$ needs to be kept to a minimum for this experiment, because thinner cells require shorter wavelengths to generate data at a useful resolution, and the cells of interest have a thickness on the order of microns. Shrinking $W$ below the diameter of the transducers could interfere with the transducers' data collection, so it cannot be meaningfully exploited within the context of these experiments. This leaves only $L$ available for manipulation. Introducing a distance between the transducers and cell allows the experimental set-up to avoid Fresnel diffraction, but also introduces some losses due to the change in transmission media \todo{be more specific here}. The magnitude of the losses depends in part on the difference in the impedances of the two media, so Rexolite (a variety of Nylon) was selected for its similar impedance to \todo{cell material} and its machinability. For this material, it was important to be able to machine a flat surface onto the faces which contact the ultrasonic transducers and the test articles, in order to keep any dust or air from getting into the path of the ultrasonic waves and introducing noise to the readings.
 
\section{Actuator}
The pneumatic piston was chosen as the actuator primarily for its relatively high precision and its relatively low cost. Other advantages include the relatively low set-up and calibration time required, its ease of control by adding an electronic or manually set pressure regulator into its air line, and its lack of any extraneous degrees of freedom. This piston in particular was chosen for having an internal double rail, which further the potential for the piston to introduce misalignments into the system. Another factor considered in choosing the particular piston were its form factor. The chosen piston has flat sides and copious mounting holes, making it convenient to mount into the apparatus. Another important consideration is range of pressure outputs. While most of the time the desired stack pressure will likely be in Cannarella's "low stack pressure" regime (0-0.5MPa) \todo{cite}, its useful to leave open the option of inducing Cannarella's "high stack pressure" regime, peaking at as much as 3.0 MPa, in order to induce lithium plating, if necessary. \todo{insert calculations} Thus, it is useful to have a piston capable of a broad range of output forces. A final useful feature was double acting, allowing the piston to instantaneouly either extend or retract.
    