\chapter{Apparatus design}

\section{Considerations}
\begin{enumerate}
    \item Frequency: resolution vs attenuation
    \begin{enumerate}
        \item Thinner cells require shorter wave lengths for useful resolution
	    \item Shorter wavelengths require higher frequencies
		\item Higher frequencies introduce greater attenuation
	\end{enumerate}
	\item Set-up: avoid near-field effects without introducing significant impedance
		\begin{enumerate} 
		    \item impedance is the result of medium transitions (e.g. the light traveling through air into a pool of water problem from physics)'
		    \item The near-field effects make the earliest readings highly unwieldy
	        \item Need to develop an apparatus that places enough distance between the transducer and the cell that the near-field effects have fallen away, but does so without introducing significant new impedances - be sure to impedance match materials and minimize number of materials
	    \end{enumerate}
	\item Pressure vs performance
	    \begin{enumerate}
		\item Need to exert enough pressure on the cell to ensure quality, consistent readings, but not so much that it materially effects battery behavior
		\item Should be adjustable
		\item Should apply a consistent pressure (<=0.5MPa) for SOC/SOH reasons, preferably light (to reduce experiment rest times) but potentially heavy if additional lithium deposition needs to be forced (see Cannarella)
		\end{enumerate}
    \item Securely and planarly mount as many as three transducers (to compare readings from different regions of cell - deposition should be asymmetric since leads are on one side)
    \item Avoid clamping to the edges: the lamination sometimes causes lil bulges
\end{enumerate}

Questions:
\begin{enumerate}
	\item How to couple transducer to cell?
	\item How to clamp cell?
	\item How to secure transducers?
    \item How to actuate set-up?
\end{enumerate}

\section{Sensors}

\section{Structure}

\section{Actuator}
Pneumatic piston was chosen for:
\begin{enumerate}
    \item relatively low cost (since air infrastructure is already present in lab)
    \item low expected set-up/calibration time
    \item relatively high precision
    \item easily controlled by controlling the pressure of its intake
    \item doesn't introduce any unnecessary degrees of freedom
\end{enumerate}

Piston selection:
\begin{enumerate}
    \item Form factor: flat sides, lots of mounting holes to conveniently and securely mount to structure
    \item Force output: Will most probably want to exert a pressure in the range of 0-0.5 MPa (Cannarella's "low stack pressure"), but want to leave the option open of exerting a stack pressure of 1.5-3.0 MPa on cell (Cannarella's "high stack pressure"), in order to induce lithium plating if necessary; if this is done across an area of ~20cm\^2. McMaster (a convenient supplier) offers pistons of the desired form factor with force ratings of up to 779lbf at 100 psi, which assuming a clamp area of 4.5 cm\^2, would result in an exerted stack pressure of about 1.7 MPa.
    \item Double-acting: will allow the piston to both push and pull itself, allowing for a more automated set-up
    \item Non-rotating: piston constructed of two parallel rods, which prevents the piston from introducing a rotational moment
    \item Stroke-length: implications ??
\end{enumerate}
    