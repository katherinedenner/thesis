% broad introduction
% when choosing what to include, assume the reader will only skip it
%   include any critical background in a subsection of the relevant chapter
\chapter{Background}

\section{Lithium Ion Fast Charge Cells}

\subsection{Lithium Ion Battery Chemistry}

\subsection{Fast Charging}

\subsection{Lithium Plating}

\subsection{These Cells}

\todo{section title}

\section{Acoustic Time of Flight (ToF) Analysis}
Recent work has shown that Acoustic Time of Flight (ToF) Analysis can show the state of charge and state of health of battery cells.

explain ToF

First, Hsieh et al \todo{cite?} performed electrochemical-acoustic time-of-flight (EAToF) experiments on batteries of various chemistries and form factors, demononstrating "strong correlation between SOC and the density distribution within a cell, as determined by acoustic measurements" \cite{TOF-STATE}. They go on to conclude that "changes in the ToF echo profiles and acoustic signal amplitudes as a function of cycle number appear to be key indicators of central phenomena occurring within the battery, including changes in intraparticle and interparticle stress and strain, as well as the formation and removal of critical surface layers", suggesting but not demonstrating that electrochemical-acoustic ToF analysis can determine a battery cell's state of health.

That assertion was demonstrated in the subsequent paper by Bhadra et al \todo{cite?} \cite{ANODE-CHAR}, which demonstrated the capacity of electrochemical acoustic time-of-flight analysis to "examine the dynamic properties of cells during discharge" "by tracking the saturation of the central echo and secondary echos, the total shift in the ToF peak position over discharge, and total transmitted signal amplitude".

This work was pushed futher by Davies et al \todo{cite?} \cite{SOC-SOH-EST}, which explored the accuracy of the EAToF method while cycling lithium-ion pouch cells over hundreds of cycles, producing "two key metrics: time of flight shift and total signal amplitude, which are the used with voltage data in a supervised machine learning technique to build a model for the state of charge (SOC) prediction". Not only did they show their model was about $99\%$ accurate for two different cell chemistries, but they showed that the model can be extended to predict state of health with similar accuracy by adding to the model the full ultrasonic waveforms at top of charge.

\todo{waves blurb}

\todo{Frensel and Fraunhofer components}

\todo{Other factors which affect $c$}
What some noise contributions are

\subsection{Prior Work}