% broad introduction
% when choosing what to include, assume the reader will only skip it
%   include any critical background in a subsection of the relevant chapter
\section{Background}
\subsection{Lithium Ion Fast Charge Cells}

\subsubsection{Lithium Ion Battery Chemistry}
Increasing demands on the energy density and form factor of rechargeable batteries created a massive and growing market for lithium ion battery cells \cite{LBST}. 
Their superior gravimetric and volumetric energy densities are due to the high cell voltages (~4V), which are made possible by the non-aqueous electrolytes of lithium cells \cite{LBST}. 
Battery cells work by passing ions back and forth between two electrodes - separated, charged materials. In its fully charged state, a battery cells charge has collected on its negatively-charged electrode (the anode). 
The ions are keen to flow to a lower energy state, and flow readily to the positively charged electrode (the cathode), shedding their electrons to the circuit in the process and powering the connected device. 
As more of the ions jump to the cathode, the energy imbalance between the two plates shrinks, until charged particles are no longer easily motivated to move from the anode to the cathode. 
In a rechargeable cell, this process is reversible, and if electrons are fed back to the cathode by reversing the circuit, ions will flow back to the anode \cite{LBST}. 

The voltage of a battery is simply its potential energy per charge, and the open-circuit voltage $V_{oc}$ of a lithium-ion cell is a function of
$$ V_{oc} = \frac{\mu _{Li(c)} - \mu _{Li(a)}}{F} $$
where $\mu_{Li(c)}$ is the lithium chemical potential of the cathode, $\mu_{Li(a)}$ is the lithium potential of the anode, and $F$ is Faraday's constant (the charge in Coulombs of one mole of electrons) \cite{LBST}.
Good charge and discharge performance requires that the "lithium insertion/extraction process should be reversible with no or minimal changes in the host structure over the entire range $x$ of lithium insertion in order to provide a good cycle life for the cell" \cite{LBST}.

\subsubsection{Lithium Plating}
The reactivity of lithium with its electrolyte can lead to considerable performance degradation \cite{HANDBOOK}. 
This is exasperated because repetitive charge-discharge cycling can permanently change the cell \cite{HANDBOOK}.
During recharging, as the lithium is forced back onto the anode, it electroplates, forming a surface layer onto the electrode which has a larger surface area than the original metal \cite{HANDBOOK}. 
Thus, charge and recharge cycling increases the surface area of the reacting surface, which correspondingly increases the reactivity \cite{HANDBOOK}. 
Aggressive charge and discharge can force enough plating to significantly increase the surface area of the electrode, thus significantly increasing the cell's reactivity.
These increases in reactivity reduce the cell's thermal stability, rendering the cell increasingly sensitive to abuse due to further cycling \cite{HANDBOOK}.

The lithium electroplating and unplating which results from cyclic recharging and discharging also causes the surface of the lithium to develop a film of the reaction products between it and the electrolyte \cite{HANDBOOK}. 
Each stripping and plating of the lithium with the reaction products' risks growing the film on the lithium's surface \cite{HANDBOOK}. 
As the film grows in size, the proportion of electrochemically unreactive lithium grows, reducing the cell's useful capacity \cite{HANDBOOK}.
Operating the cell in a way which prevents or reduces lithium passivation can prevent routine operational reduction in capacity, functionally increasing the capacity of the cell \cite{HANDBOOK}.

\subsubsection{Charging Protocols}

\subsubsection*{Stage 0: Preconditioning}

If the cell was deeply discharged, it should first be charged at a very low current, so that minimal heat is generated before the cell is in such a state that it can deal with that heat \cite{DIGIKEY}.

\subsubsection*{Stage 1: Constant-Current} 

Drive the cell at a constant current to get its voltage to 4.2V. The higher the current, the faster this charging stage happens, but also the more damage delivered to the cell \cite{TI}. Charging rates are often limited to 1C, but fast charging can allow for higher rates \cite{TI}.

\subsubsection*{Stage 2: Constant-Voltage}  

Charge with constant-voltage until the current steps down to ~0.1 Amps, at which point charging is complete \cite{DIGIKEY}. Constant-voltage charging prevents overcharge \cite{DIGIKEY}. The decreasing current required to maintain 4.2V is what creates the exponential decay shape of the current vs time plots during charging \cite{TI}.

\subsubsection{Charging Rates}
To normalize across battery capacities, charge or discharge current is often expressed as a C-rate \cite{SPECS}. 
A C-rate expresses the rate a battery at which a battery is discharged in terms of its maximum capacity; a 1C rate will fully discharge a battery in 1 hour, a C/2 rate will do so in 2 hours, and a 2C rate will discharge a battery in 1/2 hour \cite{SPECS}. 
Since the cells being investigated in this work have a roughly x Amp-hours \todo{num} capacity, a 1C rate uses \todo{num} Amps for charging, whereas a 10C rate uses \todo{num} Amps.

\subsection{Acoustic Time of Flight (ToF) Analysis}
Recent work has shown that acoustic Time of Flight (ToF) Analysis can show the state of charge and state of health of battery cells.

Acoustic ToF analysis probes a unit with continuous acoustic waves sent between two ultrasonic transducers. One transducer transmits the waves, and the other uses a pulse-echo mode to receive the waves and their reflections \cite{TOF-STATE}. Data about how long it takes the wave and its reflections to travel from the transmitter to the receiver can be used to estimate the state of charge or health of the cell \cite{TOF-STATE}. 
How long it takes for a wave to travel from one point to another depends on the ubiquitous relation 
$$\text{distance} = \text{ rate} \times \text{ time} \rightarrow \text{ time} = \frac{\text{distance}}{\text{rate}}$$

Nondestructive ultrasonic analysis measures the time of flight (how long it takes the wave to travel either back to the transmitter, or to a receiver), and the amplitude of the received signal. 
These quantities can be used to estimate the thickness of the material, $T$:
$$ T = \frac{ct}{2}$$

where $c$ is the material sound velocity, an $t$ is the duration of the flight \cite{OLYMPUS}. 
Thus, if $c$ can be assumed constant, $T$ can be measured using only $t$.
Ensuring $c$ can be assumed sufficiently constant was an important consideration in apparatus design.

The smaller the wavelength, the smaller a defect in a material that can be detected via time of flight analysis \cite{OLYMPUS}. Since wavelength and frequency are inversely related, high frequency transducers are necessary since the cell is so thin.

\subsection{Prior Work}

Hsieh et al performed electrochemical-acoustic time-of-flight (EAToF) experiments on batteries of various chemistries and form factors, demonstrating "strong correlation between SOC and the density distribution within a cell, as determined by acoustic measurements" \cite{TOF-STATE}. They go on to conclude that "changes in the ToF echo profiles and acoustic signal amplitudes as a function of cycle number appear to be key indicators of central phenomena occurring within the battery, including changes in intraparticle and interparticle stress and strain, as well as the formation and removal of critical surface layers", suggesting but not demonstrating that electrochemical-acoustic ToF analysis can determine a battery cell's state of health.

That assertion was demonstrated in the subsequent paper by Bhadra et al \cite{ANODE-CHAR}, which demonstrated the capacity of electrochemical acoustic time-of-flight analysis to "examine the dynamic properties of cells during discharge" "by tracking the saturation of the central echo and secondary echos, the total shift in the ToF peak position over discharge, and total transmitted signal amplitude".

This work was built on by Davies et al \cite{SOC-SOH-EST}, who explored the accuracy of the EAToF method while cycling lithium-ion pouch cells over hundreds of cycles, focusing on "two key metrics: time of flight shift and total signal amplitude, which are the used with voltage data in a supervised machine learning technique to build a model for the state of charge (SOC) prediction". 
In addition to showing their model was about $99\%$ accurate for two different cell chemistries, they showed that the model can be extended to predict state of health with similar accuracy by adding to the model the full ultrasonic waveforms at top of charge.